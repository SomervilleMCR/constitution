\documentclass[11pt, a4paper]{article}

% This package defines the page margins
\usepackage[top=2.0cm, bottom=2.0cm, left=2.0cm, right=2.0cm]{geometry}

% This package prevents hyphenation of words
\usepackage[none]{hyphenat}

% This package helps re-define list structure
\usepackage{enumitem}
\setlist[enumerate,1]{label=\alph*)}  % list level 1 is letters followed by )
\setlist[enumerate,2]{label=\roman*.} % list level 2 is Roman numerals followed by .
\setlist[enumerate,3]{label=\arabic*.} % list level 2 is Arabic numerals followed by .


\begin{document}





%%%%%%%%%%%%%%%%%%%%%%%%%%%%%%%%%%%%
%%%%% CONSTITUTION STARTS HERE %%%%%
%%%%%%%%%%%%%%%%%%%%%%%%%%%%%%%%%%%%

\centerline{{\Huge \textsc{Constitution}}}
\vspace{2mm}
\centerline{{\Large \textsc{of the}}}
\vspace{2mm}
\centerline{{\Large \textsc{Somerville College Middle Common Room}}}





\section{Preamble}
\label{sec:preamble}

The goals of the Middle Common Room are to provide a mutually supportive environment for its members, to represent their academic and social interests, to recognise and welcome the multi-cultural interests of the MCR, and to act as a channel of communication with the College, the University, and the community at large.

Generations of graduate Somervillians have been committed to the creation of a diverse and supportive environment for different cultures. We pride ourselves on our history of high academic standards and generous social spirit.  We strive for the inclusion of all members and oppose discrimination on any grounds. The tone of this Middle Common Room has been that of openness and inclusiveness. We embrace an atmosphere of tolerance and the expression of identity.

In this spirit, we, the graduate students of Somerville College, in order to secure and transmit to succeeding generations our academic and social heritage, do establish this Constitution for our Middle Common Room.





\section{Definitions}
\label{sec:definitions}

The terms that follow hereinafter are defined as follows:

\begin{enumerate}
	\item ``MCR'' refers to the Somerville College Middle Common Room.
    \item ``MFH'', ``The House'', or ``Margery Fry'' refers to Margery Fry Elizabeth Nuffield House.
    \item ``The Executive'' refers to the MCR Executive Committee.
	\item ``SCR'' refers to the Somerville College Senior Common Room.
	\item ``JCR'' refers to the Somerville College Junior Common Room.
	\item ``GB'' refers to the Somerville College Governing Body.
	\item ``The College'' refers to Somerville College, Oxford.
	\item ``Member'' refers to a member of the MCR as defined in Section \ref{sec:membership}.
    \item ``Notice'' refers to communications distributed on the MCR mailing list.
\end{enumerate}





\section{Name}
\label{sec:name}

\begin{enumerate}
	\item There will be a Somerville Middle Common Room.
    \item Any representative use of the MCR banner or name must be authorised by The Executive.
\end{enumerate}





\section{Membership}
\label{sec:membership}

There will be Full Members of the MCR.
There may be Associate and Honorary Members of the MCR.

\begin{enumerate}
	\item Full Members will be:
        \begin{enumerate}
      		\item Any student with a degree from any university who is currently registered in a postgraduate course at The College who is not in suspended status;
      		\item Any student at The College who is currently registered in a second undergraduate degree;
      		\item Any student at The College who is currently registered in a graduate-entry medicine degree;
      		\item Any student at The College who is currently registered for the Diploma in Legal Studies;
      		\item Any mature student at the college, having been 21 or over at the start of October in their first year.
		\end{enumerate}
    \item Full Members will retain the following rights:
    	\begin{enumerate}
            \item To vote in all MCR elections and meetings;
            \item To stand for an office of The Executive or as a representative of the MCR;
            \item To access The College, MFH, the library, and the dining hall, at the discretion of College authorities;
            \item To attend MCR events and those College events directed at the MCR.
		\end{enumerate}
	\item The following are eligible for Associate Membership:
		\begin{enumerate}
                \item Partners of current Full Members;
                \item Former full members;
                \item Suspended or lapsed full members;
                \item Those eligible for Full Membership who elect not to take Full Membership;
                \item Fourth year undergraduate students at The College;
                \item Others, as proposed by the Tutor for Graduates;
                \item Others, as proposed by The Executive.
		\end{enumerate}
    \item Associate Membership will be granted upon application to The Executive, subject to the approval of the Principal and Tutor for Graduates of the College.
    \item Approved Associate Members:
    	\begin{enumerate}
      		\item Will have access to enter The College, MFH, the library, and the dining hall, at the discretion of College authorities;
            \item Are entitled to attend MCR events;
            \item Do not have the right to vote in MCR elections or at General or Emergency meetings.
		\end{enumerate}
    \item Associate membership will expire at the end of each academic year, with the option for renewal by application and approval.
    \item There will be Honorary Members of the MCR. Honorary Membership will be granted through a Motion put to a General Meeting with proper notice and subsequently approved by a majority of Full Members present. It is customary for such a motion to include the rationale for the granting of Honorary Membership in the preamble to the motion. Honorary Membership is granted for life or until it is revoked. Honorary Membership is a sinecure.
     \item The list of Associate and Honorary members will be shared with the Porters' Lodge and the Tutor for Graduates, and a record will be kept by the Secretary of the MCR.
\end{enumerate}





\section{Removal of Associate and Honorary Membership}
\label{sec:removal_member}

Associate or Honorary Membership can be removed by a majority of the Full Members present at a General or Emergency Meeting of the MCR.





\section{The MCR Executive Committee}
\label{sec:executive}


\begin{enumerate}
	\item There will be a democratically-elected Executive consisting of:
    \begin{enumerate}
        \item The President;
        \item The Vice President;
        \item The Secretary;
        \item The Treasurer;
        \item The Social Secretary or Secretaries;
        \item The House Chair;
        \item The Welfare Officer or Officers.
    \end{enumerate}
    \item The collective responsibilities of The Executive are to:
    \begin{enumerate}
        \item Convene Executive meetings at least twice per term, at which quorum will be 50\% of The Executive;
        \item Ensure the effective coordination and efficient operation of the MCR and its activities;
        \item Facilitate the smooth transition of responsibilities from one Executive to the next;
        \item Attend all Executive Meetings, General Meetings, and Emergency Meetings, unless absence is unavoidable. Apologies for absences should be sent to the MCR Secretary at the earliest opportunity;
        \item Request and judge applications for the Barbara Craig Fund and distribute the award(s) according to Schedule 3 of this Constitution.
    \end{enumerate}
    \item The functions of the MCR President are to:
    \begin{enumerate}
        \item Ensure the smooth running and development of the MCR, through liaison with The Executive, non-Executive Representatives, and other Members;
        \item Represent and promote the interests of the MCR at both the College and University level, including attendance at all relevant committee meetings;
        \item Meet with the Principal, Treasurer, Deans, Tutor for Graduates, and other College staff when necessary;
        \item Chair Executive Meetings and arrange for a substitute in their absence;
        \item Ensure MCR activities are reported at General Meetings;
        \item Maintain and build relationships with other MCRs, and also with The College SCR and JCR.
    \end{enumerate}
    \item The functions of the MCR Vice President are to:
    \begin{enumerate}
        \item Support the President in all their functions;
        \item Act as a substitute for the President whenever empowered to do so by the President;
        \item Oversee online ballots, as per Schedule 4 of this Constitution.
    \end{enumerate}
    \item The functions of the MCR Secretary are to:
    \begin{enumerate}
        \item Give Notice of General and Emergency Meetings;
        \item Receive motions for General and Emergency Meetings;
        \item Give Notice of agenda and motions for General and Emergency Meetings;
        \item Take minutes at General, Executive, and Emergency Meetings or appoint a substitute to take minutes if unable to attend;
        \item Publish and file the minutes of General, Executive, and Emergency Meetings within ten days of the meeting;
        \item Maintain the official record of the Constitution and Schedules.
    \end{enumerate}
    \item The functions of the MCR Treasurer are to:
    \begin{enumerate}
        \item Keep the financial records and be responsible for the MCR accounts;
        \item Present a brief financial statement to The Executive at the beginning of every term, and give a complete report at the end of their term of office;
        \item Prepare a projected annual budget for the MCR, to be presented to the membership at the first General Meeting of their tenure, and prepare a balance sheet of revenue and expenses, representing a true and accurate account of the financial position of the MCR, to be distributed at each General Meeting;
        \item Ensure that the accounts are audited at the end of their term of office by at least two non-Executive members of the MCR;
        \item Meet with the College Treasurer and Accountant when necessary to discuss the financial affairs of the MCR;
        \item Represent the MCR, together with the MCR President, at meetings of the Finance Committee of The College;
        \item Report on the status of the Barbara Craig Fund once per year at a General Meeting.
    \end{enumerate}
    \item The functions of the MCR Social Secretary or Secretaries are to:
    \begin{enumerate}
        \item Coordinate social events for the MCR;
        \item Give Notice to members of upcoming events;
        \item Organise exchange dinners with other Colleges;
        \item Produce a term card by Sunday of Week One each term;
        \item Produce a programme of events for Freshers' Week and early arrivals in Michaelmas term.
    \end{enumerate}
    \item The functions of the MCR House Chair are to:
    \begin{enumerate}
        \item Welcome all new members to The House and ensure that throughout their period of residence they are provided with all relevant information concerning living in The House;
        \item Be in regular contact with relevant College staff and non-Executive Representatives in relation to the smooth running and maintenance of The House;
        \item Report on House matters and the status of the MFH fund at all General Meetings. At that time they shall speak to issues raised concerning The House, in writing if absent;
        \item Administer the MFH fund, using the money to purchase items required for the kitchens and for other purposes in consultation with the residents;
        \item Ensure that all kitchens are regularly checked and stocked with necessary provisions (for instance, washing-up liquid and clean tea towels);
        \item Be responsible for the provisions and upkeep of the common areas of The House.
    \end{enumerate}
    \item The functions of the Welfare Officer or Officers are to:
    \begin{enumerate}
        \item Attend to the welfare needs of Members and, where appropriate, refer them to suitable resources including Junior Deans, College Nurse, and University Counselling Service;
        \item Attend welfare training events;
        \item Organise and give Notice of regular welfare events;
        \item Liaise with non-Executive welfare or equalities officers, such as representatives for LGBTQ, Disabilities and Ethnic Minorities, as well as with other volunteers to ensure all welfare needs of Members are met.
    \end{enumerate}
    \item There will be non-Executive Representative positions as determined from time to time by The Executive or by the MCR through a General Meeting. Descriptions of previous representative positions will be held by the MCR Secretary. Elections for non-Executive positions, if necessary, will be overseen by The Executive with Notice given for nominations and voting. Individuals can be appointed to uncontested, non-Executive positions.
\end{enumerate}





\section{Meetings}
\label{sec:meetings}

\begin{enumerate}
	\item General Meetings:
        \begin{enumerate}
      		\item There will be General Meetings of the MCR at least twice per term, at a time arranged by The Executive.
      		\item Notice of each meeting will be given at least one week in advance.
            \item An agenda will be circulated by email at least 24 hours before the meeting.
            \item Motions should be emailed to the MCR Secretary up to 48 hours before the meeting.
            \item Quorum, for each vote, is 15 Full Members of the MCR, including at least 50\% of The Executive.
            \item Meetings will be held in the Common Room of The House unless The Executive notifies, at the time of Notice of the meeting, otherwise.
            \item Meetings will be governed as per Schedule 2.
		\end{enumerate}
    \item Emergency Meetings:
    	\begin{enumerate}
      		\item An Emergency Meeting may be called in the event of:
            	\begin{enumerate}
                	\item The majority decision of The Executive; or
                    \item A petition by 15 of the Full Members.
                \end{enumerate}
      		\item Notice of an Emergency Meeting will be given at least 3 days in advance, with the motions proposed.
            \item Motions to be considered at an Emergency Meeting will be:
            	\begin{enumerate}
                	\item Those approved by a majority decision of The Executive;
                    \item Those included in the petition for an Emergency Meeting as signed by 15 of the Full Members.
                \end{enumerate}
            \item Quorum, for each vote, is 15 Full Members of the MCR.
            \item Emergency Meetings will be held in the Common Room of The House.
            \item Emergency Meetings will be governed as per Schedule 2.
		\end{enumerate}
\end{enumerate}





\section{Elections}
\label{sec:elections}

\begin{enumerate}
	\item All Executive posts will be put for election in the second half of Hilary Term.
    \item The Executive takes office on Sunday of Week 9 in Hilary Term.
    \item The rules governing elections can be found in Schedule 1.
\end{enumerate}





\section{Removal of an Executive Member or non-Executive Representative}
\label{sec:removal_committee}

A member of The Executive or a non-Executive Representative immediately ceases to hold office if they:

\begin{enumerate}
	\item Resign; or
    \item Cease to be a Member; or
    \item Are removed by a motion passed, by at least two-thirds of Full Members present, at a General Meeting or Emergency Meeting.
\end{enumerate}





\section{Finance}
\label{sec:finance}

\begin{enumerate}
	\item The MCR will have a bank account operated by the Treasurer or such person or persons as The Executive empower.
    \item The Treasurer will present a statement of accounts at each General Meeting.
    \item All expenditure of \pounds200 or more from the MCR account will be approved either by a General Meeting or by an Online Ballot as detailed in Schedule 4.
\end{enumerate}





\section{Amendments to the Constitution and Schedules}
\label{sec:amendment}

\begin{enumerate}
	\item Resolutions to amend the Constitution or Schedules must be submitted to The Executive, who will give Notice to Members of the exact wording at least one week prior to the General Meeting at which they are to be first considered. Minor amendments to such resolutions may be made at a General Meeting, provided such changes do not substantially alter the resolution.
    \item Amendments to the Constitution or Schedules must be proposed and seconded by Full Members.
    \item Amendments to the Constitution or Schedules must have the support of a two-thirds majority of a General Meeting.
    \item Successful amendments to the Constitution or Schedules take effect upon their approval by a meeting of the GB.
\end{enumerate}





\section{Authority}
\label{sec:authority}

\begin{enumerate}
	\item Any question about the interpretation of this Constitution may be submitted to The Executive to make a decision by majority. If The Executive fails to reach a majority decision, the question of interpretation will be put to the Tutor for Graduates for a final ruling.
    \item This Constitution is subject to the approval of the GB and a copy of the Constitution should be submitted to the GB in the Hilary Term of each academic year for review.
\end{enumerate}





\section{Complaints}
\label{sec:complaints}

\begin{enumerate}
	\item Complaints about any matter pertaining to the MCR should be submitted, in the first instance, to The Executive, if deemed acceptable to the complainant. The Executive will investigate. If The Executive cannot come to resolution acceptable both to the complainant and to the individuals complained about, all parties involved should consult with the Tutor for Graduates, who will be empowered to make a final decision, and must explain said decision to all parties involved.
    \item If the complainant is dissatisfied with the ruling of the Tutor for Graduates, they may appeal in writing to the GB.
\end{enumerate}





\section{Effect}
\label{sec:effect}

All previous Constitutions of the MCR and rights derived from them are revoked and this constitution has effect as of Hilary Term 2017.





\clearpage
\setcounter{section}{0}




%%%%%%%%%%%%%%%%%%%%%%%%%%%%%%%%%%
%%%%% SCHEDULE 1 STARTS HERE %%%%%
%%%%%%%%%%%%%%%%%%%%%%%%%%%%%%%%%%

\centerline{{\Huge \textsc{Schedule 1: Election Procedures}}}
\vspace{2mm}
\centerline{{\Large \textsc{To Accompany the Constitution of the}}}
\vspace{2mm}
\centerline{{\Large \textsc{Somerville College Middle Common Room}}}





\section{Election of a Chief Returning Officer}
\label{sec:cro}

\begin{enumerate}
	\item There will be a Chief Returning Officer (CRO).
    \item At the first General Meeting of Hilary Term, the MCR will elect a CRO to supervise the election of The Executive.
    \item The CRO shall be a Member, either Full or Associate.
    \item Notice will be given of the election of a CRO by the Secretary at least one week prior to the General Meeting.
    \item Nominations for the position of CRO will close at 09:00 on the morning of the General Meeting.
    \item Members will elect the CRO by secret ballot at the General Meeting.
    \item In the event that there is no candidate for the position of CRO at the close of nominations, a candidate will be selected at the General Meeting.
    \item In the event that there is a single candidate for CRO, they will be deemed appointed by the MCR.
    \item In the event of a tie between candidates for CRO, a candidate will be chosen by a method consented to by all tied candidates.
\end{enumerate}





\section{Duties of the Chief Returning Officer}
\label{sec:cro_duties}

\begin{enumerate}
	\item Duties of the CRO will be:
    	\begin{enumerate}
        	\item The overall supervision of The Executive elections;
            \item Giving Notice of The Executive election as laid out in this Schedule;
            \item Opening and closing nominations for The Executive elections as laid out in this Schedule;
            \item Organising hustings for all declared candidates, which will take place the day prior to elections;
            \item Organising voting;
            \item Announcing Executive election results;
            \item Organising secondary elections, where necessary, for Executive positions.
        \end{enumerate}
    \item The CRO cannot be a candidate for The Executive in the elections they are to supervise.
    \item At no point during the campaign shall the CRO endorse or denounce candidates for The Executive.
    \item In the event of the resignation of a CRO, if time permits, a new CRO will be selected in a manner similar to the procedure above. If this is not possible, the  Executive elections will be supervised by the Tutor for Graduates or their representative.
\end{enumerate}





\section{Nominations}
\label{sec:nominations}

\begin{enumerate}
	\item Nominations must be open for at least one week.
    \item Nominations will be opened by the CRO  by Notice.
    \item Candidates who have been nominated and seconded will be contacted by the CRO and will send their acceptance of their nomination by email to the CRO.
    \item Nominees for Social Secretary may run in teams of up to three people.
    \item Nominees for Welfare Officer may run in teams of up to two people.
\end{enumerate}





\section{Voting}
\label{sec:voting}

\begin{enumerate}
	\item Voting will take place by confidential and anonymous internet ballot, from 08:00 to 20:00 on the Sunday chosen for elections.
    \item The CRO will distribute voting instructions by email at least 12 hours before voting begins.
    \item The CRO will be provided with a list of all eligible voters by The Executive before the start of voting. The CRO will consult this list to ensure only eligible voters are given voting access.
    \item The CRO will post the results of the election before 23:59 on the day of voting.
\end{enumerate}





\section{Secondary Elections}
\label{sec:secondary_elections}

\begin{enumerate}
	\item In the event that there are no candidates for a given Executive position at the close of nominations, or in the event that the winner of a given election is RON (re-open nominations), or in the event that there is a tie between two candidates (including between a candidate and RON),  a second election will be held.
    \item This secondary election will be supervised by the CRO, and will be run in a manner similar to the procedure above.
    \item In the event of a tie between candidates in a Secondary Election, a candidate will be chosen by a method consented to by all candidates.
\end{enumerate}





\section{By-Elections}
\label{sec:by_elections}

\begin{enumerate}
	\item In the event that a member of The Executive resigns or is removed from office during their term of office, the CRO will hold a by-election.
    \item The CRO will prepare an election similar to that of an Executive Election.
    \item In the event that no CRO is incumbent when a by-election is required, The Executive will appoint a CRO for the by-election.
\end{enumerate}





\section{Continued Vacancies}
\label{sec:continued_vacancies}

If there remains no candidate for a position on The Executive following a Secondary Election, then the duties of that positions are to be distributed by the members of The Executive.





\section{Appeals}
\label{sec:appeals}

Complaints about the operation any election should be put to the Tutor for Graduates in writing.





\clearpage
\setcounter{section}{0}




%%%%%%%%%%%%%%%%%%%%%%%%%%%%%%%%%%
%%%%% SCHEDULE 2 STARTS HERE %%%%%
%%%%%%%%%%%%%%%%%%%%%%%%%%%%%%%%%%

\centerline{{\Huge \textsc{Schedule 2: Conduct of Meetings}}}
\vspace{2mm}
\centerline{{\Large \textsc{To Accompany the Constitution of the}}}
\vspace{2mm}
\centerline{{\Large \textsc{Somerville College Middle Common Room}}}





\section{Chair}
\label{sec:chair}

\begin{enumerate}
	\item There will be a Chair of each General or Emergency meeting of the MCR.
    \item The President will be the Chair of General or Emergency meetings. In their absence, their nominated representative will serve as Chair.
    \item The Chair of a General or Emergency Meeting will:
    	\begin{enumerate}
        	\item Follow the agenda as posted;
            \item Decide the priority of speaking;
            \item Keep good order within the meeting;
            \item Tally the votes on each question put and announce the result to the membership present.
        \end{enumerate}
\end{enumerate}





\section{Agenda}
\label{sec:agenda}

\begin{enumerate}
	\item Notice of the agenda for each meeting will be given by the MCR Secretary at least 24 hours prior to the meeting.
    \item The Agenda will include opportunity for any other business to be discussed.
\end{enumerate}





\section{Motions}
\label{sec:motions}

\begin{enumerate}
	\item Motions must be submitted to the MCR Secretary no later than 48 hours before the meeting.
    \item Motions must be proposed by one Full Member and seconded by another Full Member. The proponent and the seconder will be denoted on the motion.
    \item The proponent of a Motion has first right to speak in relation to the Motion they proposed.
\end{enumerate}





\section{Voting}
\label{sec:meetings_voting}

\begin{enumerate}
	\item Voting on Motions will normally be by show of hands.
    \item Any Full Member may request a roll-call vote.
    \item At the time of voting, any Full Member can move a motion for the vote to take place by secret ballot. Such a motion requires a majority of Full Members.
    \item A majority or a vote of two-thirds does not include abstentions, but only those votes cast for or against the motion.
    \item If amendments to the Constitution are passed, notice of the changes must be given to Members within ten days.
\end{enumerate}





\clearpage
\setcounter{section}{0}




%%%%%%%%%%%%%%%%%%%%%%%%%%%%%%%%%%
%%%%% SCHEDULE 3 STARTS HERE %%%%%
%%%%%%%%%%%%%%%%%%%%%%%%%%%%%%%%%%

\centerline{{\Huge \textsc{Schedule 3: Barbara Craig Fund}}}
\vspace{2mm}
\centerline{{\Large \textsc{To Accompany the Constitution of the}}}
\vspace{2mm}
\centerline{{\Large \textsc{Somerville College Middle Common Room}}}





\section{The Barbara Craig Fund}
\label{sec:bcf}
Barbara Craig, a former Principal of Somerville, established a fund for the MCR from which the MCR can make small grants to graduate students to assist with expenses relating to their studies or personal development.





\section{Procedure}
\label{sec:procedure}

\begin{enumerate}
	\item The Treasurer will inquire about the amount the Barbara Craig Fund has provided by Week 3 in Hilary Term.
	\item Applications for awards will be advertised by Notice, and will be open for at least two weeks.
    \item Applications will be due by Sunday of Week 9 in Hilary Term.
    \item Awards will be announced prior to the start of Trinity Term.
	\item There shall be an Awarding Committee made up of those Executive members who have not applied for an award.
	\item If the Awarding Committee has fewer than five members, volunteers shall be sought by Notice.  Such volunteers must not have applied for an award, and may be either Full or Associate Members.
	\item If the Awarding Committee has fewer than five members by Sunday of Week 9 in Hilary Term, the Tutor for Graduates, or their nominated representative, shall join the Awarding Committee.
    \item The Awarding Committee will distribute the funds in any way they see fit, subject to guidelines as provided by The College and the MCR Executive.
    
\end{enumerate}





\section{Applications to the Barbara Craig Fund}
\label{sec:applications}

\begin{enumerate}
    \item Full Members may apply for grants from the Barbara Craig Fund for any purpose they feel contributes to their academic development, to the life of the MCR, or to Somerville College in general.
    \item Applications may be for, but are not limited to:
    	\begin{enumerate}
        	\item Travel expenses;
            \item Books;
            \item Software;
            \item Organising theatre productions;
            \item Participating in sporting events.
        \end{enumerate}
    \item Applicants must demonstrate that they have applied to other sources for financial support, if applicable, especially in the case of travel and book grants for which Somerville College has other sources of funding.
    \item Applications will be submitted via email to the President, who will anonymise them prior to  assessment.
    \item Committee members must declare any conflicts of interest when returning their judgements and should recuse themselves at the request of President.
\end{enumerate}





\clearpage
\setcounter{section}{0}




%%%%%%%%%%%%%%%%%%%%%%%%%%%%%%%%%%
%%%%% SCHEDULE 4 STARTS HERE %%%%%
%%%%%%%%%%%%%%%%%%%%%%%%%%%%%%%%%%

\centerline{{\Huge \textsc{Schedule 4:}}}
\vspace{2mm}
\centerline{{\Huge \textsc{Online Ballots and Referenda}}}
\vspace{2mm}
\centerline{{\Large \textsc{To Accompany the Constitution of the}}}
\vspace{2mm}
\centerline{{\Large \textsc{Somerville College Middle Common Room}}}





\section{Online Voting}
\label{sec:online_voting}

\begin{enumerate}
	\item A decision requiring approval of a General Meeting may instead be decided by means of online ballot, following the same regulations as meeting decisions including:
    	\begin{enumerate}
        	\item Notice period;
            \item Quorum.
        \end{enumerate}
    \item Online ballots are intended to allow for decisions to be made during times that make the calling of General Meetings difficult or impossible; for instance, during vacations.
    \item The Vice President will oversee online ballots.
\end{enumerate}





\section{Referenda}
\label{sec:referenda}

\begin{enumerate}
	\item A motion will be put to a referendum following:
    	\begin{enumerate}
        	\item A two-thirds majority vote at a General Meeting; or
            \item Presentation of a petition to The Executive signed by 15 Full Members.
        \end{enumerate}
	\item Action on the motion contested in a referendum should be halted until referendum procedures are completed.
    \item Debate on referenda will occur at the next General Meeting or Emergency Meeting, whichever comes first.
    \item Voting will follow via online ballot, after circulation of minutes of the meeting at which debate occurred.
\end{enumerate}




\end{document}
